\documentclass[11pt]{article}
\usepackage[margin=1in]{geometry}
\usepackage[T1]{fontenc}
\usepackage[utf8]{inputenc}
\usepackage{booktabs}
\usepackage{longtable}
\usepackage{xcolor}
\usepackage{array}
\usepackage{lmodern}
\usepackage{hyperref}
\usepackage{graphicx}

\hypersetup{colorlinks=true, linkcolor=black, urlcolor=blue}

\begin{document}

\begin{center}
{\LARGE \textbf{Mineral Report}} \\
\vspace{0.4em}
{\Large Chrysocolla} \\
\vspace{0.2em}
Generated (UTC): 2026-02-22T21:25:49.423509232+00:00
\end{center}

\vspace{1em}

\begin{center}
\includegraphics[width=0.36\textwidth]{ image.jpg }
\end{center}
\vspace{0.6em}


\section*{Context}
\begin{tabular}{>{\raggedright\arraybackslash}p{0.28\textwidth} p{0.67\textwidth}}
\textbf{Audience} & technical geologist \\
\textbf{Purpose} & exploration briefing \\
\textbf{Site Context} & pilot drill campaign \\
\end{tabular}

\section*{Physical and Chemical Snapshot}
\begin{tabular}{>{\raggedright\arraybackslash}p{0.28\textwidth} p{0.67\textwidth}}
\textbf{Mineral Family} & Silicates \\
\textbf{Description} & Blue-green, earthy to slightly waxy copper-bearing silicate coating and filling vugs on a brown, iron-oxide-rich host rock. The massive, porous texture and turquoise-green coloration are consistent with chrysocolla commonly formed in the oxidized zone of copper deposits, sometimes intergrown with quartz/opal and minor malachite. \\
\textbf{Formula} & (Cu,Al)2H2Si2O5(OH)4·nH2O \\
\textbf{Hardness (Mohs)} & 2.50 \\
\textbf{Hardness Band} & soft \\
\textbf{Density (g/cm\textsuperscript{3})} & 2.20 \\
\textbf{Density Band} & light \\
\textbf{Crystal System} & Microcrystalline \\
\textbf{Color} & blue-green to green with brown host rock \\
\textbf{Streak} & White to pale blue-green \\
\textbf{Luster} & Dull to waxy/earthy \\
\textbf{Dominant Element} & O (52.0 wt\%) \\
\end{tabular}

\vspace{0.8em}
\textbf{Interpretive Summary}
\begin{quote}
For technical geologist and the pilot drill campaign context, Chrysocolla is classified as soft with light density behavior. The chemistry is led by O (52.0 wt\%), supporting exploration briefing decisions.
\end{quote}

\section*{Major Elements}
\begin{longtable}{p{0.42\textwidth} p{0.42\textwidth}}
\toprule
\textbf{Element} & \textbf{Weight \%} \\
\midrule
\endhead

O & 52.00 \\

Si & 18.00 \\

Cu & 16.00 \\

Fe & 9.00 \\

H & 3.00 \\

Al & 2.00 \\

\bottomrule
\end{longtable}

\section*{Recommendations}
\begin{enumerate}

\item Prioritize samples of Chrysocolla where O enrichment is strongest.

\item Validate breakage and weathering rates early, as softer material can bias grade control.

\item Combine XRD with geochemistry to avoid over-reliance on density-based separation.

\item Archive this report against 'exploration briefing' objectives for reproducible decision records.

\end{enumerate}

\section*{Field Notes}
Chrysocolla is compositionally variable and often mixed with quartz/opal; hardness can range \textasciitilde{}2–4 depending on silica content. Brown areas likely limonite/goethite or iron-stained rock; green may include minor malachite. Streak typically pale blue-green to white.

\end{document}