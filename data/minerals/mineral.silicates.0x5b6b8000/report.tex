\documentclass[11pt]{article}
\usepackage[margin=1in]{geometry}
\usepackage{fontspec}
\usepackage{xeCJK}
\usepackage{polyglossia}
\usepackage{booktabs}
\usepackage{longtable}
\usepackage{xcolor}
\usepackage{array}
\usepackage{hyperref}
\usepackage{graphicx}

\setmainfont{Noto Serif}
\setsansfont{Noto Sans}
\setmonofont{Noto Sans Mono}
\newfontfamily\arabicfont[Script=Arabic,Scale=MatchLowercase]{Noto Naskh Arabic}
\newfontfamily\hindifont[Script=Devanagari,Scale=MatchLowercase]{Noto Serif Devanagari}
\newfontfamily\devanagarifont[Script=Devanagari,Scale=MatchLowercase]{Noto Serif Devanagari}

\setCJKmainfont{Noto Serif CJK SC}
\setCJKsansfont{Noto Sans CJK SC}



\setdefaultlanguage{english}


\hypersetup{colorlinks=true, linkcolor=black, urlcolor=blue}

\begin{document}

\begin{center}
{\LARGE \textbf{ 矿物报告 }} \\
\vspace{0.4em}
{\Large 硅铍石} \\
\vspace{0.2em}
生成时间 (UTC): 2026-02-23T23:27:30.480610142+00:00
\end{center}

\vspace{1em}

\begin{center}
\includegraphics[width=0.36\textwidth]{ image.jpg }
\end{center}
\vspace{0.6em}


\section*{ 上下文 }
\begin{tabular}{>{\raggedright\arraybackslash}p{0.28\textwidth} p{0.67\textwidth}}
\textbf{ 受众 } & 技术地质人员 \\
\textbf{ 目的 } & 勘查简报 \\
\textbf{ 现场背景 } & 试点钻探活动 \\
\end{tabular}

\section*{ 物理与化学概览 }
\begin{tabular}{>{\raggedright\arraybackslash}p{0.28\textwidth} p{0.67\textwidth}}
\textbf{ 族 } & 硅酸盐 \\
\textbf{ 描述 } & 半透明白色至无色的硅铍石,呈块状、柱状晶形,具细微纵向条纹,光泽为玻璃光泽至略带油脂光泽。硅铍石为铍硅酸盐,常产于花岗伟晶岩与热液脉中;当晶面不完整时,标本可呈块状或短柱状。 \\
\textbf{ 化学式 } & Be2SiO4 \\
\textbf{ 硬度 (Mohs) } & 7.50 \\
\textbf{ 硬度等级 } & 很硬 \\
\textbf{ 密度 (g/cm3) } & 2.97 \\
\textbf{ 密度等级 } & 中等 \\
\textbf{ 晶系 } & 三方晶系 \\
\textbf{ 颜色 } & 无色至白色(半透明) \\
\textbf{ 条痕 } & 白色 \\
\textbf{ 光泽 } & 玻璃光泽至略带油脂光泽 \\
\textbf{ 主导元素 } & O (48.8 wt\%) \\
\end{tabular}

\vspace{0.8em}
\textbf{ 解释性总结 }
\begin{quote}
面向技术地质人员并结合试点钻探活动场景,硅铍石被判定为很硬,密度表现为中等。其化学组成以O为主(48.8 wt\%),可支持勘查简报相关决策。
\end{quote}

\section*{ 主要元素 }
\begin{longtable}{p{0.42\textwidth} p{0.42\textwidth}}
\toprule
\textbf{ 主要元素 } & \textbf{ 质量百分比 } \\
\midrule
\endhead

O & 48.80 \\

Si & 31.30 \\

Be & 19.90 \\

\bottomrule
\end{longtable}

\section*{ 建议 }
\begin{enumerate}

\item 优先采集 硅铍石 中 O 富集最明显的样品。

\item 使用耐磨工具,并上调粉碎能耗估算。

\item 将 XRD 与地球化学结合,避免过度依赖基于密度的分选。

\item 请将本报告归档到“勘查简报”目标下,以保留可复现的决策记录。

\end{enumerate}

\section*{ 备注 }
鉴定基于用户提供的背景与一般外观;许多白色半透明矿物在照片中外观相近。建议通过硬度(应能刻划石英)、无解理(贝壳状/不平坦断口)及比重约\textasciitilde{}3.0进行确认。

\end{document}